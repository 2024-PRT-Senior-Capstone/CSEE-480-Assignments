\subsection{Overall Architecture}
\subsubsection{Overall}
The Raspberry Pi Model 4B collects data from three devices—an imaging sensor, GPS, and a current clamp. 
The imaging sensor counts the number of people inside the PRT car, the GPS hat tracks the current location of the PRT car, and the current clamp monitors the voltage and current being drawn by the PRT car. 
These three devices transfer their data to the Raspberry Pi. 
Once the Pi is stopped for about 5 seconds and has a Wi-Fi connection to WVU Encrypted, the data transfers to a SQL database hosted on a Google Cloud server instance. 
Next, a web application retrieves the data from the SQL database to visualize it, providing graphs and charts to analyze relationships between the data. 
The overall architecture is in Figure 1 and Figure 2, which represents the system functionality from top to bottom.

\begin{center}
    \includegraphics[width=0.5\textwidth]{Overall_Architecture.png}\\
    Fig 2. Overall Architecture
\end{center}

\begin{center}
    \includegraphics[width=0.5\textwidth]{System_Functionality.png}\\
    Fig 3. System Functionality
\end{center}

\subsubsection{Hardware}
Four devices fit onto the Raspberry Pi for the collection of data. 
The connections are shown below in Figure 4. 
The Pi communicates with the GPS for real-time location. 
The antenna plugs into the GPS and realizes the data collection. 
The current clamp collects voltage levels and current in an AC environment, recording it to the Pi. 
A debug screen displays the GPS status, IP address, and the current AC reading for debugging purposes. 
The debug screen displays this information to ensure the hardware components work as intended. 
Lastly, the imaging sensor collects snapshots of people for detecting the number of people in a PRT car. 
The snapshots go to the Raspberry Pi, where the Pi will use an algorithm to count the number of people in that snapshot.

\begin{center}
    \includegraphics[width=0.5\textwidth]{Hardware_Architecture.png}\\
    Fig 4. Hardware Architecture
\end{center}

\subsubsection{Software}
When the Raspberry Pi obtains a Wi-Fi connection through WVU Encrypted and is at a standstill for 5 seconds or more, data goes to Google Cloud. 
Figure 5 represents the process of data transfer once Wi-Fi connects. 
The Google Cloud hosts a SQL database that sorts and stores all the metadata from the Pi system. 
A web application uses this sorted data from the SQL database, providing visualization through charts and graphs to highlight relationships in the data. 
The web application provides opportunities for identifying potential risks in the system, such as irregular AC currents or voltage on a specific PRT car.

\begin{center}
    \includegraphics[width=0.5\textwidth]{Software_Architecture.png}\\
    Fig 5. Software Architecture
\end{center}

\subsection{System Design}
The solution that has been designed is an information system, a combination of hardware and software aimed at collecting and processing data and distributing it to a database over the internet.
Further designing is required in collaboration with the PRT engineers to understand the physical implementation restrictions.

\subsubsection{Timeline}
The PRT Capstone project timeline fits into two semesters. 
The whole timeline is shown below in Figure 6 as a Gantt chart. 
The first semester focuses on the hardware portion of the project: connecting the GPS, current clamp, and LiDAR to the Raspberry Pi and testing each for successful integration and creation of a demo. 
Once the integration is successful, the device will be tested by connecting it to the PRT. 
Ending the first semester, after successful testing, the Pi is placed into an enclosure. 
The second semester focuses on the software portion: configuring data to send to the Google Cloud, which hosts a SQL database. 
Google Cloud will format the data before placing it into the SQL database, where it's sorted and stored. 
Next, a web application interfaced with Grafana will visualize the sorted data with charts and graphs.

\begin{adjustwidth}{-75pt}{-100pt}
    \begin{ganttchart}[
        vgrid,
        newline shortcut=false,
        y unit chart=.75cm,
        y unit title=0.5cm,
        title height=1.0,
        bar height=0.6,
        group height=0.3,
        group progress label node/.append style={below=100pt},
        group progress label font=\color{white!0!white}\sffamily,
        group label node/.append style=%
        bar progress label node/.append style={below=100pt},
        bar progress label font=\color{white!0!white}\sffamily,
        bar label node/.append style=%
        {align=left}
        ]{1}{32}
        \gantttitle{Spring 2024}{16}
        \gantttitle{Fall 2024}{16} \ganttnewline
    
        \gantttitle{Hardware}{14}
        \gantttitle{Link}{4}
        \gantttitle{Software}{14} \ganttnewline
    
        \ganttgroup[progress=78]{Phase 1}{1}{4} \ganttnewline
        \ganttbar[progress=100]{
            Draft architectures
        }{1}{1} \ganttnewline
        \ganttbar[progress=100]{
            Acquire Raspberry Pi
        }{2}{2} \ganttnewline
        \ganttbar[progress=67]{
            Acquire Sensors
        }{2}{2} \ganttnewline
        \ganttbar[progress=67]{
            Test hardware
        }{3}{3} \ganttnewline
        \ganttbar[progress=67]{
            Gather preliminary data
        }{3}{3} \ganttnewline
        \ganttbar[progress=67]{
            Configure hardware
        }{4}{4} \ganttnewline
    
        \ganttgroup[progress=67]{Phase 2}{5}{9} \ganttnewline
        \ganttbar[progress=100]{
            Connect GPS 
        }{5}{5} \ganttnewline
        \ganttbar[progress=100]{
            Connect Current Clamp
        }{6}{6} \ganttnewline
        \ganttbar[progress=0]{
            Connect LiDAR
        }{7}{7} \ganttnewline
        \ganttbar[progress=67]{
            Test
        }{8}{8} \ganttnewline
        \ganttbar[progress=67]{
            Create Demo
        }{9}{9} \ganttnewline
    
        \ganttgroup[progress=0]{Phase 3}{10}{14} \ganttnewline
        \ganttbar[progress=0]{
            Connect to PRT
        }{10}{11} \ganttnewline
        \ganttbar[progress=0]{
            Power Hardware
        }{12}{12} \ganttnewline
        \ganttbar[progress=0]{
            Hardware enclosure
        }{13}{14} \ganttnewline
    
        \ganttgroup[progress=0]{Phase 4}{15}{18} \ganttnewline
        \ganttbar[progress=0]{
            Draft data architectures
        }{15}{16} \ganttnewline
        \ganttbar[progress=0]{
            Configure data dumps
        }{17}{18} \ganttnewline
        
        \ganttgroup[progress=0]{Phase 5}{19}{25} \ganttnewline
        \ganttbar[progress=0]{
            Format data
        }{19}{20} \ganttnewline
        \ganttbar[progress=0]{
            Configure database
        }{21}{22} \ganttnewline
        \ganttbar[progress=0]{
            Data generalization
        }{23}{25} \ganttnewline
    
        \ganttgroup[progress=0]{Phase 6}{26}{32} \ganttnewline
        \ganttbar[progress=0]{
            Data analysis scripts
        }{26}{28} \ganttnewline
        \ganttbar[progress=0]{
            Front-end data visualization
        }{29}{32} \ganttnewline
    
        \ganttlink{elem0}{elem17}
        \ganttlink{elem7}{elem17}
        \ganttlink{elem13}{elem17}
        \ganttlink{elem17}{elem20}
        \ganttlink{elem17}{elem24}
    \end{ganttchart}
\end{adjustwidth}
    
\begin{center}
    Fig 6. Timeline as a Gantt Chart
\end{center}

\subsubsection{Design Choices}
Cost, privacy, safety, and protection are important factors to consider throughout the implementation of this design. 
With a 200-dollar budget, making inexpensive decisions is crucial. 
Although, LiDAR can be an expensive device that would consume 200 dollars alone \cite{linkedinEvaluateEnvironmental}. 
There are also concerns with eye protection due to the use of lasers \cite{linkedinEvaluateEnvironmental}. 
LiDAR using class 1 lasers is considered safe for human eyes and must be necessary for this project \cite{liaLaserHazard}. 
Furthermore, finding the right one that is inexpensive is critical. 
However, if such cases do not exist, a small camera for a Raspberry Pi would be a good alternative because they are cheap \cite{amazon}. 

To mitigate privacy concerns in this project, encryption at rest, in transit, and use is necessary. 
The camera, if used, will take snapshots of the PRT car at specific intervals to determine the number of people inside. 
Furthermore, using Wi-Fi with this device instead of having a cellular connection keeps potential hackers from attempting to access the device when the Raspberry Pi is disconnected from Wi-Fi and collecting data.

Increasing the safety of passengers with the device is vital for preventing or detecting harm done by the device itself or the PRT car. 
As noted earlier, the LiDAR sensor must have class 1 lasers to deter eye damage to passengers. 
In the event of faulty wires on the PRT car, the current clamp connected to the device will detect potential hazards and display them on the web application. 
Any irregularities in the current clamp data are identifiable in the graphs and charts created in the web application.

Protecting the Raspberry Pi and its connected components must be a high priority. 
The device location on the PRT car is essential to prevent any passengers from breaking the device. 
Placing the device inside the  PRT car–under the hood–prevents this destruction from a passenger. 
Other factors like heat and electricity could also break this device. 
To counteract these factors, a conformal coating, a type of electronic water-proofing, will shield the device circuitry. 
To avoid catastrophic power supply issues, the Pi will be battery-powered if the Raspberry Pi system is entirely separate from the inner workings of the PRT car. 
If the Pi connects to the PRT car's inner wiring, it will use a boost-buck converter to drop the PRT car’s voltage down to 5v. 
This also protects the Pi against low and high voltage scenarios


\subsection{Risks}
As with any project, there are risks involved in implementation of this project. 
For the PRT cars, environmental factors like rainy weather conditions or elevated heat or cold temperatures may pose a threat to the data collection system or impact the system performance.

In order for the project to be a convenient solution for the data collection problem, the solution needs to be easily maintainable. 
Ensuring maintainability is crucial post-installation for system accessibility. 

Security is paramount, with data encryption and secure transmission protocols mitigating risks. 

Additionally, the choice of LiDAR sensors directly affects passenger eye health, highlighting the importance of careful selection.


\subsection{Test Plans}
To ensure that our solution meets the requirements specified, multiple stages and techniques of testing will be conducted. 
As we are currently developing proofs of concept, our testing for this stage involves gathering data and observing that it is accurate and useful. 
It is during this stage that we must observe the practicality of our initial ideas, noting what works and what doesn’t.

\noindent\textbf{Phase 2:}\\
\indent Once the proof of concept for the entire design can be established, the next stage of testing will involve implementing a fully functional prototype in an environment that resembles the goal environment. 
For this project, that environment is the PRT. 
We must simulate the PRT, to the best of our ability, to create a demonstration of the product. 
In this environment we will observe how our solution performs, measuring it against our predetermined goals. 

\noindent\textbf{Phase 2-3:}\\
\indent Then, the solution must be stress tested to determine how the system will perform under different circumstances, some being situations that are not ideal. 
For example, simulating extreme temperatures by determining the effectiveness of the solution on a temperature scale that is based on 2023 temperatures for Morgantown. 
Last year, the high was 94F and the low was 7F. 
Therefore, by demonstrating the ability of the solution to perform on a temperature scale of 0F - 100F, we can justify implementation.

In addition to environmental stress testing, a goal of the stress testing would be to find a number of passengers for which the capacity counter fails.
By doing this, we can properly communicate the ranges of which the solution works.

\noindent\textbf{Phase 3:}\\
\indent Once determined that the implementation is ready for use on the PRT, the data collection will begin on just a single car. 
Should the system perform to standards, it could then be implemented on all cars in the PRT System.


Overall, the testing process for the PRT data collection system involves several key stages. 
Initially, proof of concept testing verifies the feasibility and practicality of the system's initial ideas, ensuring accuracy and usefulness of gathered data. 
Prototype implementation follows, simulating the PRT environment to assess real-world performance against predetermined goals. 
Stress testing then evaluates system resilience under various conditions, identifying potential weaknesses for improvement. 
Finally, field testing on a single PRT car validates functionality, usability, and reliability before potential system-wide implementation. 
Each stage serves as a critical checkpoint, iteratively refining the solution to meet project goals efficiently.

\subparagraph{GPS Data Visualization Proof of Concept:}
\textit{Objective:} Using the GPS on the PRT to visualize a path as proof of concept.

\textit{Procedure:} We will be attaching the GPS to the PRT vehicle and record its path through the GPS data. We will then use the test run to help us visualize the path traveled by the PRT vehicle.

\textit{Outcome to be achieved:} We should be able to visualize the route from the GPS data and this will then demonstrate the effectiveness of the data collected by the GPS data during its route.

\subparagraph{Current Clamp Testing with Space Heater:}
\textit{Objective:} Verify the functionality of the current clamp on a space heater to show its effectiveness as a subtitle for the internal PRT

\textit{Procedure:} We will attach the current clamp to a space heater and record its power usage. We will take and analyze the data received from this and measure the power of consumption.

\textit{Outcome to be achieved:} We should be able to retrieve data from the space heater that shows the current clamp is accurate and working perfectly.
