The solution that has been designed is an information system, a combination of hardware and software aimed at collecting and processing data and distributing it to a database over the internet.
Further designing is required in collaboration with the PRT engineers to understand the physical implementation restrictions.

\subsubsection{Timeline}
The PRT Capstone project timeline fits into two semesters. 
The whole timeline is shown below in Figure 6 as a Gantt chart. 
The first semester focuses on the hardware portion of the project: connecting the GPS, current clamp, and LiDAR to the Raspberry Pi and testing each for successful integration and creation of a demo. 
Once the integration is successful, the device will be tested by connecting it to the PRT. 
Ending the first semester, after successful testing, the Pi is placed into an enclosure. 
The second semester focuses on the software portion: configuring data to send to the Google Cloud, which hosts a SQL database. 
Google Cloud will format the data before placing it into the SQL database, where it's sorted and stored. 
Next, a web application interfaced with Grafana will visualize the sorted data with charts and graphs.

\begin{adjustwidth}{-75pt}{-100pt}
    \begin{ganttchart}[
        vgrid,
        newline shortcut=false,
        y unit chart=.75cm,
        y unit title=0.5cm,
        title height=1.0,
        bar height=0.6,
        group height=0.3,
        group progress label node/.append style={below=100pt},
        group progress label font=\color{white!0!white}\sffamily,
        group label node/.append style=%
        bar progress label node/.append style={below=100pt},
        bar progress label font=\color{white!0!white}\sffamily,
        bar label node/.append style=%
        {align=left}
        ]{1}{32}
        \gantttitle{Spring 2024}{16}
        \gantttitle{Fall 2024}{16} \ganttnewline
    
        \gantttitle{Hardware}{14}
        \gantttitle{Link}{4}
        \gantttitle{Software}{14} \ganttnewline
    
        \ganttgroup[progress=78]{Phase 1}{1}{4} \ganttnewline
        \ganttbar[progress=100]{
            Draft architectures
        }{1}{1} \ganttnewline
        \ganttbar[progress=100]{
            Acquire Raspberry Pi
        }{2}{2} \ganttnewline
        \ganttbar[progress=67]{
            Acquire Sensors
        }{2}{2} \ganttnewline
        \ganttbar[progress=67]{
            Test hardware
        }{3}{3} \ganttnewline
        \ganttbar[progress=67]{
            Gather preliminary data
        }{3}{3} \ganttnewline
        \ganttbar[progress=67]{
            Configure hardware
        }{4}{4} \ganttnewline
    
        \ganttgroup[progress=67]{Phase 2}{5}{9} \ganttnewline
        \ganttbar[progress=100]{
            Connect GPS 
        }{5}{5} \ganttnewline
        \ganttbar[progress=100]{
            Connect Current Clamp
        }{6}{6} \ganttnewline
        \ganttbar[progress=0]{
            Connect LiDAR
        }{7}{7} \ganttnewline
        \ganttbar[progress=67]{
            Test
        }{8}{8} \ganttnewline
        \ganttbar[progress=67]{
            Create Demo
        }{9}{9} \ganttnewline
    
        \ganttgroup[progress=0]{Phase 3}{10}{14} \ganttnewline
        \ganttbar[progress=0]{
            Connect to PRT
        }{10}{11} \ganttnewline
        \ganttbar[progress=0]{
            Power Hardware
        }{12}{12} \ganttnewline
        \ganttbar[progress=0]{
            Hardware enclosure
        }{13}{14} \ganttnewline
    
        \ganttgroup[progress=0]{Phase 4}{15}{18} \ganttnewline
        \ganttbar[progress=0]{
            Draft data architectures
        }{15}{16} \ganttnewline
        \ganttbar[progress=0]{
            Configure data dumps
        }{17}{18} \ganttnewline
        
        \ganttgroup[progress=0]{Phase 5}{19}{25} \ganttnewline
        \ganttbar[progress=0]{
            Format data
        }{19}{20} \ganttnewline
        \ganttbar[progress=0]{
            Configure database
        }{21}{22} \ganttnewline
        \ganttbar[progress=0]{
            Data generalization
        }{23}{25} \ganttnewline
    
        \ganttgroup[progress=0]{Phase 6}{26}{32} \ganttnewline
        \ganttbar[progress=0]{
            Data analysis scripts
        }{26}{28} \ganttnewline
        \ganttbar[progress=0]{
            Front-end data visualization
        }{29}{32} \ganttnewline
    
        \ganttlink{elem0}{elem17}
        \ganttlink{elem7}{elem17}
        \ganttlink{elem13}{elem17}
        \ganttlink{elem17}{elem20}
        \ganttlink{elem17}{elem24}
    \end{ganttchart}
\end{adjustwidth}
    
\begin{center}
    Fig 6. Timeline as a Gantt Chart
\end{center}

\subsubsection{Design Choices}
Cost, privacy, safety, and protection are important factors to consider throughout the implementation of this design. 
With a 200-dollar budget, making inexpensive decisions is crucial. 
Although, LiDAR can be an expensive device that would consume 200 dollars alone \cite{linkedinEvaluateEnvironmental}. 
There are also concerns with eye protection due to the use of lasers \cite{linkedinEvaluateEnvironmental}. 
LiDAR using class 1 lasers is considered safe for human eyes and must be necessary for this project \cite{liaLaserHazard}. 
Furthermore, finding the right one that is inexpensive is critical. 
However, if such cases do not exist, a small camera for a Raspberry Pi would be a good alternative because they are cheap \cite{amazon}. 

To mitigate privacy concerns in this project, encryption at rest, in transit, and use is necessary. 
The camera, if used, will take snapshots of the PRT car at specific intervals to determine the number of people inside. 
Furthermore, using Wi-Fi with this device instead of having a cellular connection keeps potential hackers from attempting to access the device when the Raspberry Pi is disconnected from Wi-Fi and collecting data.

Increasing the safety of passengers with the device is vital for preventing or detecting harm done by the device itself or the PRT car. 
As noted earlier, the LiDAR sensor must have class 1 lasers to deter eye damage to passengers. 
In the event of faulty wires on the PRT car, the current clamp connected to the device will detect potential hazards and display them on the web application. 
Any irregularities in the current clamp data are identifiable in the graphs and charts created in the web application.

Protecting the Raspberry Pi and its connected components must be a high priority. 
The device location on the PRT car is essential to prevent any passengers from breaking the device. 
Placing the device inside the  PRT car–under the hood–prevents this destruction from a passenger. 
Other factors like heat and electricity could also break this device. 
To counteract these factors, a conformal coating, a type of electronic water-proofing, will shield the device circuitry. 
To avoid catastrophic power supply issues, the Pi will be battery-powered if the Raspberry Pi system is entirely separate from the inner workings of the PRT car. 
If the Pi connects to the PRT car's inner wiring, it will use a boost-buck converter to drop the PRT car’s voltage down to 5v. 
This also protects the Pi against low and high voltage scenarios
