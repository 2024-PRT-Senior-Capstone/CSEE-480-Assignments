\subsection{PRT Background and History}
The PRT, or Personal Rapid Transit, has been a main source of transportation for the students of West Virginia University since 1975. The project was government-funded in order to research more methods of transportation and was developed by Boeing. Since its completion, the PRT has serviced about 83 million students and averages around 15 thousand students per day, with up to an additional 10 thousand on days with football games \cite{Booth_2007}. In order to construct the PRT tracks, former industrial engineering professor Samy Elias was tasked with overseeing the development of the tracks in the 1970’s. Originally, the tracks only connected four stations, Walnut, Downtown, Engineering, and Towers, and had 45 PRT cars to service the rides. However, they decided to extend the tracks in July 1979 to include a destination on the Health Sciences campus as well as add 28 additional PRT cars \cite{Booth_2007}. 

Prior to the completion of the PRT, classes were required to be scheduled that allowed 2 hours of transportation between campuses due to traffic. After the construction of the PRT, the transportation time between campuses lowered considerably, allowing schedules to be tightened and more enrollment of students \cite{Trenkner_2011}.  In order to allow the PRT to run successfully, a control room was built in order to monitor the PRT cars as well as the PRT platform. Typically, it is run by two people, watching the cameras installed at each PRT station as well as monitoring the state of currently running PRT cars \cite{Trenkner_2011}. While the tracking isn’t fully accurate, it allows for the control station to notice when there is a PRT car that has broken down mid transport.

\subsection{Prior Works}
The collection and use of GPS data for transportation purposes has proven beneficial in past projects. In 2006, a project developed in Beijing for their Bus Rapid Transit (BRT) Systems used GPS data to calibrate a traffic flow simulation software, VISSIM, to improve the operational efficiency of Beijing’s BRT systems. The project emphasizes the use of GPS data and its use for algorithmic analysis \cite{Yu_Yu_Chen_Wan_Guo_2006}. The data collected from their GPS is not dissimilar to the data that WVU’s PRT would produce. The PRT is a railed transport, while the BRT is road-based and deals with traffic signals, intersections, and turning ratios at each intersection. Although the data differs from our design, it achieves similar goals–optimizing the system for better performance.
	
It is crucial to determine the number of passengers on public transportation in cases of emergency. A study in 2023 covers an option for counting passengers by using passive infrared sensors at the entrance and exit of a public bus. The sensors connect to a wireless sensor network for real-time monitoring \cite{Jurak_Osman_Sikirić_Šimunović_2023}. These sensors enable quick detection and response time while saving energy consumption and cost by removing cabling from the equation. This design differs from the PRT, as both doors on it function as an entrance and an exit. There are no designated entrances and exits, removing the possibility of using infrared sensors to count passengers as this study did previously. Instead, the PRT capstone design aims to use a LiDAR sensor to determine the number of passengers in each cart. The LiDAR sensor would be used in a wireless sensor network similar to the study described above, connecting to a Raspberry Pi microcontroller.

\subsection{Justification}
Previous groups of CSEE480 and 481 students have been able to develop a Windows-based user interface to communicate with the Raspberry Pi, but have not yet been able to implement a fully integrated solution with data acquisition from a robust suite of sensors. The lack of ability to analyze data associated with PRT use is detrimental to understanding the system and its points of inefficiency. The PRT has diligently served the students of West Virginia University for nearly 50 years. At that time, technology has advanced drastically. Now, there is more reason than ever to invest time and energy into capturing data for analysis to provide future Mountaineers with an efficient and reliable form of rapid personal transportation.

Our proposed solution is inexpensive, requiring only already-made components. The novelty of the design lies in how we combine components to solve the problem elegantly. Using a LiDAR sensor allows for real-time high accuracy and speed to identify the number of passengers on board the PRT while maintaining minimal costs. The sensor is also largely automated, saving time and resources by reducing the need for human intervention for data collection. A Raspberry Pi as a microcontroller provides a cost-effective solution for transmitting data to PRT engineers at the central maintenance station and incorporating devices like LiDAR and GPS. When connected to a GPS, the Raspberry Pi will determine its precise location in real-time and provide flexibility with the configuration of a GPS.

As well as technical improvements, there are possibilities to improve the safety of passengers. Currently, when a vehicle experiences an issue, the PRT Engineers may not know with absolute certainty which car has a problem and where it is. Adding features to track capacity and location allow for those monitoring the system to be aware should they need to respond to an emergency situation.
