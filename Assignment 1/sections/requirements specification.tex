\subsection{Functional Requirements}
\begin{enumerate}
    \item Analyze location and power data.
    \item Display data in a single-page user interface.
    \item Track vehicle occupancy.
    \item Track vehicle location and time to complete a ride.
    \item Supply data to PRT officials to increase the efficiency of the PRT system.
\end{enumerate}

\subsection{Engineering Requirements}
\begin{enumerate}
    \item Equip PRT cars with a Raspberry Pi 4 model B. 
    \item Attach additional sensors (GPS, Current Clamp).
    \item Devices should be non-visible to the occupants.
    \item Data will be transmitted to the database while at PRT platforms.
    \item Data will be transmitted wirelessly.
    \item Data transmitted will be encrypted. 
    \item Data will be stored in an SQL database.
    \item The database will be kept securely to ensure the confidentiality of data.
    \item Use of Infrared or Lidar to determine vehicle occupancy.
    \item Use data collected to make predictions.
\end{enumerate}

\subsection{Marketing Requirements}
\begin{enumerate}
    \item An automated design for monitoring GPS, AC power, and passenger amount.
    \item Low-cost devices compared to dedicated hardware.
    \item Reduce congestion at PRT stations.
    \item Protect PRT carts from combustion due to higher-than-normal voltage.
    \item Prevent carts from unintentional shutdowns.
\end{enumerate}

\subsection{Mapping of Marketing Requirements to Engineering Requirements}
Most of the requirements for our project's marketing and engineering aspects are directly proportional to each other. The specifications on hardware and devices in both sets of requirements require automation and wireless communication, which satisfies both sets. However, our engineering requirements will not directly cause lower congestion times, protection from combustion, and unintentional shutdowns. This is because we will collect the data from our project and send it to the PRT engineering team, who can then make decisions to improve the PRT based on the data.

\subsection{Engineering and Marketing Requirements trade-off chart}
\begin{center}
    Table 1: Engineering \& Marketing Rade-off Chart w/ Legend
    \begin{tabularx}{\textwidth}{|X|X|X|}
        \hline
        \checkmark : Meets Requirement & - : Low & + : High \\
        \hline
    \end{tabularx}
    \begin{tabularx}{\textwidth}{|X|c|c|c|c|c|} 
        \hline
        & Speed & Performance & Cost & Security & Real-time Efficiency\\
        \hline
        Data transport (TCP) & - & + & & + & \\
        \hline
        Data encryption (TLS) & + & + & & + & \\
        \hline
        Data storage (micro SD) & + & & & & \\
        \hline
        GPS & & & - & & \\
        \hline
        Wireless Modem & & & & & \checkmark \\
        \hline
        SIM Card (LTE) & - & - & -\textbf{/mo} & + & \checkmark \\
        \hline
        SIM Card (4G) & + & + & +\textbf{/mo} & + & \checkmark \\
        \hline
        Wi-Fi & ++ & & - & - & - \\
        \hline
        AC power sensing (Current Clamp) & & & & & \\
        \hline
        Raspberry Pi & \checkmark & \checkmark & - & & \\
        \hline
        LiDAR & + & + & +/- & & \\
        \hline
    \end{tabularx}
    \begin{flushleft}
        Note: SIM Card cost is an ongoing cost encurred monthly
    \end{flushleft}
\end{center}

\subsection{Competitive Benchmarks}
The product we hope to design meets a niche need. Because of that, there are few to no direct competitors. There are related systems worldwide, but their data analysis systems are proprietary information. Our closest competitor is the group working on the same project in CSEE481. This group is ahead in design by one semester. Their goal is to implement their proposed design and gather data from testing. While it would be in the best interest of this capstone project to collaborate, if we were to consider the 481 group to be in competition with us, we would need to ensure that our design was novel in some meaningful way. Most importantly, our solution would need to be cheaper, easier to implement, and provide rich insights about the data gathered.

\subsection{Various applicable constraints and standards}
\begin{itemize}
    \item \textbf{Legal:} This project will have no negative impact on students riding the PRT, so there will not be any legal constraints for our PRT project.
    \item \textbf{Privacy:} There isn't going to be any personal information collected on students riding the PRT. Instead, we will use LiDAR to determine car capacity.
    \item \textbf{Sustainability:} This device can be used for several years to get data from the PRT. The device will also have secure housing to prevent outside forces from damaging it, increasing its sustainability.
    \item \textbf{Economic:} Any sensors that will attach to the Raspberry Pi for this project or in the future should be inexpensive and within WVU’s budget.
\end{itemize}

\subsection{Broader Requirements and Constraints}
This project will not have much effect on other industries, as this project is focused on improving and collecting data for WVU 's PRT system, and will be used entirely for and within WVU’s PRT system. In terms of maintenance of the device over the course of its usage it would not require a lot of human interference other than some software updates to keep the device to date, so overall it should be simple to keep maintenance of the device.