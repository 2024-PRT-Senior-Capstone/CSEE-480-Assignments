The Personal Rapid Transit (PRT) system is a staple of West Virginia University’s Morgantown campus. The PRT is a relatively efficient system, moving thousands of students directly from their start position to their end destination across campus daily \cite{about-prt}. While the PRT, in its current state, serves WVU’s students appropriately, the system is not without its faults.

\subsection{Problem Statement}
Despite a movement rate of 15,000 students per day, there are times when commuters are left waiting at a platform as empty cars leave or certain stations are without vehicles for people to board \cite{about-prt}. The PRT can be improved to avoid some of these scenarios. Equipping cars with sensors to track data such as power consumption, GPS location, and passenger capacity allows said data to provide insights into how the PRT can be more efficient in transportation and energy consumption.
\subsection{Functional Specifications}
The solution lies in equipping the PRT vehicles with a telematics device. This device will allow for onboard data collection. The proposed solution and its necessary functions are listed below:
\begin{enumerate}
    \item Equip a PRT car with a Raspberry Pi 4 model B.
    \item The Raspberry Pi should be in an enclosure with a footprint that does not exceed the limitations presented by the PRT team.
    \item The enclosure should fit in PRT cars so as not to be seen by the occupants.
    \item Collect PRT car movement via GPS tracking.
    \item Collect PRT energy consumption via a current clamp.
    \item Collect occupancy via Lidar or Infrared.
    \item Sync data from the car to the database at PRT stations.
    \item Transmitted data is encrypted.
    \item Analyze collected data to generalize PRT car behaviors.
    \item Store data in an SQL database.
    \item Present data through a single-page front-end display
\end{enumerate}

\subsection{Constraints}
As development goes on, there are many aspects to consider. The nature of the project places dimensional constraints on the solution. The PRT car should prioritize its surface area for the movement of passengers. Therefore, the data collection solution should not fill seating or standing space from riders, implying that the Raspberry Pi board and appropriate housing will need to maintain a small footprint.

As well as dimensional constraints, there are time constraints. The PRT must be fully available to service students during its working hours. Our solution must not affect the function of the PRT as it operates. The data collection will remain an add-on for easy removal for service upgrades. Completely independent of the PRT car and system.

A topic to keep in mind during the design phase is cybersecurity. Our system must communicate securely because of the location data we will collect. Underlying cybersecurity is the need for rider privacy and safety, as the project should not allow outside forces to hijack the PRT housing the device. Components of the project regarding occupancy should avoid controversial surveillance methods, preferably finding another option entirely.

One final major constraint arises out of the disconnectedness of the PRT cars. Any kind of wireless communication we include will need to be added by us. We must keep wireless communication in mind when implementing our system.
