Due to the lack of realization of this capstone project in previous years, our group can be the first to implement this necessary solution. The lack of ability to analyze data associated with PRT use is detrimental to understanding the system and its points of inefficiency. The PRT has diligently served the students of West Virginia University for nearly 50 years. Throughout that time, technology has advanced drastically. Now, there is more reason than ever to invest time and energy into capturing data for analysis to provide future Mountaineers with an efficient and reliable form of rapid personal transportation.

Our proposed solution is inexpensive, requiring only already-made components. The novelty of the design lies in how we combine components to solve the problem elegantly. Using a Raspberry Pi as a microcontroller provides a cost-effective solution for transmitting data to PRT engineers at the central maintenance station and incorporating devices like LiDAR and GPS. When connected to a GPS, the Raspberry Pi will determine its precise location in real-time and provide flexibility with the configuration of a GPS. A LiDAR sensor allows for real-time, high-accuracy identification of passengers on board the PRT while maintaining minimal costs.

As well as technical improvements, there are possibilities to improve the safety of passengers. Currently, when a vehicle experiences an issue, the PRT Engineers may not know with 100% certainty which car has a problem and where it is. Adding features to track capacity and location allows for the PRT Engineers to monitor the system. In the case of an emergency, they will be aware of necessary details in order to respond quickly and appropriately.
