The PRT, or Personal Rapid Transit, has been a main source of transportation for the students of West Virginia University since 1975. The project was government-funded in order to research more methods of transportation and was developed by Boeing. Since its completion, the PRT has serviced about 83 million students and averages around 15 thousand students per day, with up to an additional 10 thousand on days with football games \cite{Booth_2007}. In order to construct the PRT tracks, former industrial engineering professor Samy Elias was tasked with overseeing the development of the tracks in the 1970’s. Originally, the tracks only connected four stations, Walnut, Downtown, Engineering, and Towers, and had 45 PRT cars to service the rides. However, they decided to extend the tracks in July 1979 to include a destination on the Health Sciences campus as well as add 28 additional PRT cars \cite{Booth_2007}. 

Prior to the completion of the PRT, classes were required to be scheduled that allowed 2 hours of transportation between campuses due to traffic. After the construction of the PRT, the transportation time between campuses lowered considerably, allowing schedules to be tightened and more enrollment of students \cite{Trenkner_2011}.  In order to allow the PRT to run successfully, a control room was built in order to monitor the PRT cars as well as the PRT platform. Typically, it is run by two people, watching the cameras installed at each PRT station as well as monitoring the state of currently running PRT cars \cite{Trenkner_2011}. While the tracking isn’t fully accurate, it allows for the control station to notice when there is a PRT car that has broken down mid transport.