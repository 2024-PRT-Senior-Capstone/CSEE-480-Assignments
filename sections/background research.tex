\subsection{PRT Background and History}
The Personal Rapid Transit System (PRT) is a transportation system that was designed exclusively for West Virginia University students in the early 1970s. The 69 electric car system has traversed more than 35 million miles of track while serving an estimated 83 million riders throughout its lifetime. The PRT system was introduced as a solution to address transportation challenges that had arisen for students regarding traffic congestion and parking. During its initial phase of development, the PRT exclusively serviced the route between the Walnut and Evansdale stations. However, in the late 1970s, the network expanded to encompass the Health Sciences Center, enhancing connectivity across the campus. This expansion provided students with swift access to various parts of the university and surrounding areas, drastically reducing travel times. With this extended route, students can efficiently traverse the city in mere minutes. The journey between the farthest stations, Health Sciences and Walnut, now spans just over 11 minutes in total \cite{about-prt}.

The PRT has met the need for efficient student mobility between campuses, moving 15,000 people daily on average between the 5 stations (Walnut, Beechurst, Evansdale, Towers, and Health Science) since its opening in 1975. This statistic represents a remarkable improvement in transportation efficiency compared to traditional modes such as buses or personal vehicles \cite{about-prt}.

\subsection{Prior Works}
The collection and use of GPS data for transportation purposes has been used in past projects. In 2006, a project developed in Beijing for their Bus Rapid Transit (BRT) Systems used GPS data to calibrate a traffic flow simulation software, VISSIM, to improve the operational efficiency of Beijing’s BRT systems. The project emphasizes the use of GPS data and its use for algorithmic analysis \cite{Yu_Yu_Chen_Wan_Guo_2006}. The data collected from their GPS is not dissimilar to the data that WVU’s PRT would produce. The PRT is a railed transport, while the BRT is road-based and deals with traffic signals, intersections, and turning ratios at each intersection. Although the data is different in comparison to the PRT capstone, it still achieves similar goals–optimizing the system for better performance.

It is important to determine the number of passengers on public transportation to reduce the amount of traffic congestion at each station. A study in 2023 covers an option for counting passengers by using passive infrared sensors at the entrance and exit of a public bus. The sensors connect to a wireless sensor network for real-time monitoring \cite{Jurak_Osman_Sikirić_Šimunović_2023}. This enables quick detection, and response time while saving energy consumption and cost by removing cabling out of the equation. This design differs from the PRT, as both doors on it function as an entrance and an exit. There are no designated entrances and exits, removing the possibility of using infrared sensors to count passengers as this study did previously. Instead, the PRT capstone aims to use a LiDAR sensor to count passengers in each cart. The LiDAR sensor would be used in a wireless sensor network similar to the study described above, connecting to a Raspberry Pi microcontroller.

\subsection{Justification}
Due to the lack of realization of this capstone project in the past, our group can be the first to implement this necessary solution. The lack of ability to analyze data associated with PRT use is detrimental to the understanding of the system and its points of inefficiency. The PRT has diligently served the students of West Virginia University for nearly 50 years. In that time, technology has advanced drastically. Now, there is more reason than ever to invest time and energy into capturing data for analysis to provide future Mountaineers with an efficient and reliable form of rapid personal transportation.
	
Our proposed solution is inexpensive, requiring only already-made components. The novelty of the project lies in the way that we combine components to solve the problem elegantly. The use of a LiDAR sensor allows for real-time high accuracy and speed to identify the number of passengers on board the PRT while maintaining minimal costs. The sensor is also largely automated, saving time and resources by reducing the need for human intervention for data collection. A Raspberry Pi as a microcontroller provides a cost-effective solution for transmitting data to PRT engineers at the central maintenance station and incorporating devices like LiDAR and GPS. When connected to a GPS, the Raspberry Pi will determine its precise location in real time and provide flexibility with the configuration of a GPS.
