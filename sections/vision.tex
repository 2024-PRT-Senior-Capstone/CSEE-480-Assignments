The Personal Rapid Transit system - better known as the PRT - is a staple of West Virginia University’s Morgantown campus. The PRT is a relatively efficient system, moving thousands of students directly from their start position to their end destination across campus daily \cite{about-prt}. While the PRT in its current state serves WVU’s students appropriately, the system is not without its downfalls.

\subsection{Problem Statement}
Despite a movement rate of 15,000 students per day, there are times when commuters are left waiting at a platform as empty cars leave or certain stations are left without vehicles for people to board \cite{about-prt}. The PRT can be improved to avoid some of these scenarios. If cars were equipped with the ability to track data such as power consumption, GPS location, and passenger capacity, analyzing that data could provide insights into how the PRT can be more efficient in terms of transportation and energy expenditure.

\subsection{Functional Specifications}
The solution lies in the equipment of the PRT vehicles with a telematics device. This device will allow for onboard data collection. The proposed solution and its necessary functions are described below.
\begin{enumerate}
    \item Equip a PRT car with a Raspberry Pi 4 model B.
    \item The Raspberry Pi should be in an enclosure with a footprint that does not exceed the limitations presented by the PRT team.
    \item The enclosure should fit in PRT cars so as not to be seen by the occupants.
    \item Collect PRT car movement via GPS tracking.
    \item Collect PRT energy consumption via a current clamp.
    \item Collect occupancy via Lidar or Infrared.
    \item Sync data from the car to the database at PRT stations.
    \item Data transmitted will be encrypted.
    \item Analyze collected data to generalize PRT car behaviors.
    \item Store data in an SQL database.
    \item Present data in a single-page front-end display.
\end{enumerate}

\subsection{Constraints}
As development moves forward, there are many considerations to take into account. The nature of the project places dimensional constraints on the solution. The PRT car should prioritize its surface area for the movement of passengers, therefore the data collection solution should not take seating or standing space from riders. This implies that the Raspberry Pi board and appropriate housing will need to maintain a small footprint.

As well as dimensional constraints, there are time constraints. The PRT must be available to service students during its normal working hours. Our solution must not affect the function of the PRT as it already operates. The data collection will remain as an add-on, remaining easily removable for service or upgrades and completely independent of the PRT car and system.

A topic to keep in mind during the design phase of the project is cybersecurity. Our system must communicate securely because of the location data we are going to collect. Underlying cybersecurity is the need for rider privacy. Components of the project regarding occupancy should avoid controversial methods of surveillance, preferably finding another option entirely.

One final major constraint arises out of the disconnectedness of the PRT cars. Any kind of wireless communication we wish to include will need to be added ourselves. We must keep this fact in mind as we plan our system and the communication requirements.
