The Personal Rapid Transit system - better known as the PRT - is a staple of West Virginia University’s Morgantown campus. The PRT is a relatively efficient system, moving thousands of students [prt source!!!!!!!!!!] directly from their start position to their end destination across campus daily. While the PRT in its current state serves WVU’s students appropriately, the system is not without its downfalls.

\subsection{Problem Statement}
Despite a movement rate of 15,000 students per day [same prt source!!!!!!!!], there are times when commuters are left waiting at a platform as empty cars leave or certain stations are left without vehicles for people to board. The PRT can be improved to avoid some of these scenarios. If cars were equipped with the ability to track data such as power consumption, GPS location, and passenger capacity, analyzing that data could provide insights into how the PRT can be more efficient in terms of transportation and energy expenditure.

\subsection{Functional Specifications}

\subsection{Constraints}